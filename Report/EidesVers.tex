\begin{otherlanguage}{ngerman}
\addchap*{Eidesstattliche Erklärung}

\vspace*{5mm}

\thispagestyle{empty}

\begin{flushleft}
\begin{tabular}[h]{p{60mm}l p{60mm}l}
\textbf{Name:} Schwörer 			&\textbf{Vorname:} Lukas\\
\textbf{Matrikel-Nr.:} 65283		&\textbf{Studiengang:} Mechatronik\\
\end{tabular}
\end{flushleft}

\vspace*{11mm}

Hiermit versichere ich, \textbf{Lukas Schwörer}, an Eides statt, dass ich die vorliegende Bachelorarbeit

an der \textbf{University of Halmstad}

mit dem Titel \textbf{„Hard metrology of the human visual perception“}

selbständig und ohne fremde Hilfe verfasst und keine anderen als die angegebenen Hilfsmittel benutzt habe. Die Stellen der Arbeit, die dem Wortlaut oder dem Sinne nach anderen Werken entnommen wurden, sind in jedem Fall unter Angabe der Quelle kenntlich gemacht.\\

Ich habe die Bedeutung der eidesstattlichen Versicherung und prüfungsrechtlichen Folgen (\S 23 Abs. 3 des allg. Teils der Bachelor-SPO der Hochschule Aalen) sowie die strafrechtlichen Folgen (siehe unten) einer unrichtigen oder unvollständigen eidesstattlichen Versicherung zur Kenntnis genommen.\\

\vspace*{10mm}
\Large\textbf{Auszug aus dem Strafgesetzbuch (StGB)}


\normalsize\textbf{\S 156 StGB} Falsche Versicherung an Eides Statt
Wer von einer zur Abnahme einer Versicherung an Eides Statt zuständigen Behörde eine solche Versicherung falsch abgibt oder unter Berufung auf eine solche Versicherung falsch aussagt, wird mit Freiheitsstrafe bis zu drei Jahren oder mit Geldstrafe bestraft.

\vspace*{25mm}


\rule[-0.2cm]{5cm}{0.5pt} \hspace*{30mm}\rule[-0.2cm]{5cm}{0.5pt}
\newline
Ort, Datum\hspace*{61.85mm}Unterschrift

\end{otherlanguage}