\chapter*{Kurzfassung}
\addcontentsline{toc}{chapter}{Kurzfassung}
\label{kurzfassung}
\begin{otherlanguage}{ngerman}
    
Dieses Projekt beschäftigt sich mit der Entwicklung, dem Aufbau, dem Testen und Qualifizieren einer Elektronischen Leitspindel (ELS). Dises Projekt wurde der Universität von mir vorgeschlagen.
Sein Ziel ist es ein System zu entwickeln, dass das Getriebe in einer konventionellen Drehbank ersetzt und die Rotation der Leitspindel zu der Rotation der Hauptspindel synchronisiert. Die ELS muss
fähig sein mit der Rotation der Haupspindel mitzuhalten während einer konventionellen Drehbearbeitung mit unterschiedlichen Drehzahlen und Vorschüben. Zusätzlich muss es möglich sein präzise metrische und
Imperische Gewinde zu drehen.\\

Das elektro-mechanische System der ELS besteht aus einem Encoder der die Position der Haupspindel ausliest und einem Servo-Motor, der die Position der Leitspindel kontrolliert. Ein Microcontroller
verarbeitet die vom Encoder gesammelten Informationen und bestimmt die korrekte Position des Servo-Motors.\\

Um ein einfaches Ändern und Entfernen von Features zu ermöglichen und das Verhalten des Systems vorherzusagen, muss die Entwicklung der ELS Modellbasiert durchgeführt werden. Dieses Modell muss alle
Komponenten des reellen Systems beinhalten, eingeschlossen Spindel, Encoder, Microcontroller und Servo-Motor. Zum Vergleich sollte auch ein System mit einem konventionellen Getriebe modelliert werden.

\end{otherlanguage}
