\chapter*{Kurzfassung}
\addcontentsline{toc}{chapter}{Kurzfassung}
\label{kurzfassung}
\begin{otherlanguage}{ngerman}

    Dieses Projekt beschäftigt sich mit der Entwicklung, dem Aufbau, dem Testen und Qualifizieren einer Elektronischen Leitspindel (ELS). Dieses Projekt wurde der Universität von mir vorgeschlagen, da es aufgrund der Covid-19 Pandemie nur begrenzt möglich war praktische Projekte durchzuführen. Sein Ziel ist es ein System zu entwickeln, dass das Getriebe in einer konventionellen Drehbank ersetzt und die Rotation der Leitspindel zu der Rotation der Hauptspindel synchronisiert. Die ELS muss fähig sein, mit der Rotation der Hauptspindel mitzuhalten während einer konventionellen Drehbearbeitung mit unterschiedlichen Drehzahlen und Vorschüben. Zusätzlich muss es möglich sein, präzise metrische und imperische Gewinde herzustellen.\\
 
    Das elektromechanische System der ELS besteht aus einem Encoder, der die Position der Hauptspindel ausliest und einem Servomotor, der die Position der Leitspindel kontrolliert. Ein Mikrocontroller
    verarbeitet die vom Encoder gesammelten Informationen und bestimmt die korrekte Position des Servomotors.\\
    Um ein einfaches Ändern und Entfernen von Features zu ermöglichen und das Verhalten des Systems vorherzusagen, muss die Entwicklung der ELS Modellbasiert durchgeführt werden. Dieses Modell muss alle Komponenten des reellen Systems beinhalten, eingeschlossen der Spindel, des Encoders, dem Mikrocontroller und dem Servomotor. Zum Vergleich sollte auch ein System mit einem konventionellen Getriebe modelliert werden.
    
\end{otherlanguage}
